\chapter{Implementation and Results}
This chapter explores the development and outcomes of the Bicycle Management System, focusing on its effective tracking functionality, simple and accessible interface and suitability for various use cases, while also considering existing challenges and future improvements for better dependability.
\end{center}

%This is for reference only. Delete before finalization

\section{Implementation}
The Bicycle Management System is designed to monitor and manage bicycle usage efficiently through a digital interface. Core components include a user registration module, database connectivity (using MySQL) and a responsive frontend. Backend functionalities are implemented using Spring Boot, while MySQL handles data storage and retrieval.
Users can register, log in and track bicycle usage or availability. Admins can manage bicycle records, user details  and usage history. The system ensures smooth data flow between frontend and backend, supporting real-time updates and interactions, making it suitable for campus or community bicycle programs.\cite{3.1}

\section{Performance Analysis}
The Bicycle Management System delivers effective and organized control over bicycle access and tracking. It ensures smooth performance through optimized database queries and secure login systems.\\
The system is scalable and flexible, with the ability to integrate additional features like QR scanning or GPS tracking. Challenges may arise from connectivity issues or database load under high usage, but these can be mitigated through optimization and error handling.
The project combines reliability with user convenience, making it a practical solution for sustainable transport management in smart campuses or eco-friendly communities.\cite{3.2}

\section{Results and Discussion}
The Bicycle Management System effectively manages and monitors bicycle usage through a digital platform. Testing confirmed smooth operation of core features such as user registration, login authentication and real-time bicycle tracking. The system performed reliably under normal usage conditions, with minor delays observed only under high network load or database congestion, indicating areas for optimization.\\
The integration of MySQL with Spring Boot ensured stable data handling, while the frontend provided a responsive and user-friendly experience. Admin functionalities, such as bicycle allocation and usage history monitoring, worked as intended without critical issues.\\
Overall, the system proved to be reliable, efficient, and adaptable. With potential upgrades like GPS integration or mobile app support, the project can be further enhanced for broader usage in campuses, communities, or rental services, supporting eco-friendly transportation solutions.\cite{3.3}\\

