\chapter{Engineering Standards and Mapping}
%This is for reference only. Delete before finalization
The Bicycle Management System follows engineering standards by ensuring secure data management, reliable performance and user-friendly access, with proper design, testing and user input to meet key functional and security needs.

\section{Social and Environmental Sustainability}
The Bicycle Management System contributes to social and environmental sustainability by promoting eco-friendly transportation and encouraging healthier, more active lifestyles. By making bicycle access and tracking more organized and efficient, it supports communities in reducing their carbon footprint and traffic congestion.
This system enhances mobility for students, employees and community members, offering a cost-effective and sustainable alternative to motorized transport. It is especially beneficial in campuses or urban areas where short-distance travel is common. Additionally, the digital nature of the system supports inclusive access and convenience, allowing users to locate and use bicycles with minimal effort.
However, its impact depends on accessibility, technological literacy and infrastructure. Limited digital access or lack of user awareness could affect its effectiveness. Overreliance on automated systems might also reduce personal responsibility in managing shared resources.\\
Overall, the Bicycle Management System aligns with modern sustainability goals, encouraging green habits and efficient transport solutions, while requiring thoughtful implementation to ensure broad accessibility and long-term societal benefit.
\cite{4.1}


\subsection{Impact on Life}
\textbf {Convenient Mobility:} The Bicycle Management System simplifies transportation by allowing users to easily locate, access and manage bicycles without the need for manual tracking. This promotes a more organized and accessible commuting experience, especially in campuses, offices and public areas.\\
\textbf{Time Efficiency:}With real-time data and digital access, users can quickly find available bicycles and avoid delays. This system is especially helpful in busy environments where time-saving is essential.\\
\textbf{Safety and Accessibility:}Organized bicycle access reduces clutter and potential hazards in shared spaces. In emergency or crowded situations, users can rely on the system to find safe and immediate transport options.\\
\textbf{Challenges:}Limited infrastructure, such as lack of dedicated bike lanes or storage areas, can reduce effectiveness. Some users may face difficulties adapting to the digital platform, especially those less familiar with technology. In addition, overdependence on automated systems may reduce users’ initiative in managing shared resources manually. The system may also need further enhancement to manage large-scale usage or multi-location operations.\cite{4.1.1}

\subsection{Impact on Society \& Environment}
The Bicycle Management System has a significant impact on both society and the environment. It encourages the use of bicycles as a primary mode of transport, promoting healthier lifestyles and reducing dependency on fuel-powered vehicles. This contributes to improved public health and decreased air pollution in urban and institutional environments.\\
Socially, it enhances mobility for individuals across different backgrounds, especially those who may not afford private transportation. By offering organized access and real-time tracking, it supports inclusivity and convenience, especially in educational institutions or urban communities.\\
Environmentally, the system supports sustainable transportation goals by reducing carbon emissions and traffic congestion. However, digital infrastructure and electronic device usage carry environmental considerations, such as energy consumption and e-waste from outdated systems. Proper management of digital waste and the use of energy-efficient servers and systems are essential for minimizing long-term environmental impact.\cite{4.1.2}
\subsection{Ethical Aspects}
The Bicycle Management System raises several ethical considerations including accessibility, data privacy, environmental impact and social equity. By offering an organized and efficient transportation method, the system supports inclusion and independence for a wide user base.
However, concerns about user data privacy must be addressed, particularly in systems that store personal information or track usage behavior. Secure handling, transparency in data processing and consent-based access are key ethical responsibilities.
From an environmental standpoint, the development and maintenance of the system should aim to minimize waste and power usage. Using sustainable software practices and durable hardware can reduce environmental harm.
Furthermore, the pricing and availability of such systems must consider social fairness. High costs or restricted access could disadvantage certain communities. Ethical deployment ensures that the technology remains accessible and beneficial across different socioeconomic groups without fostering inequality or overdependence on digital systems.\cite{4.1.3}
\subsection{Sustainability Plan}A sustainability plan for the Bicycle Management System incorporates environmental, social and economic goals throughout its development and deployment lifecycle. From the design phase, efforts should be made to use cloud services with low carbon footprints and efficient database systems to reduce energy use.
The backend infrastructure should be optimized for minimal energy consumption, while the frontend should be lightweight and accessible on low-end devices. Economically, offering multiple access levels free or subsidized options for students or low-income users helps ensure broader adoption.\\
Socially, promoting awareness campaigns and user education can drive responsible and effective use of the system. Long-term maintenance and updates should be planned with modular designs and scalable architecture to extend the system’s lifespan and reduce unnecessary upgrades or replacements.
At the end-of-life stage, decommissioning the system should include responsible disposal of physical devices and secure deletion of user data. Together, these steps ensure that the Bicycle Management System aligns with long-term sustainability and user satisfaction.\cite{4.1.4}

%\section

\section{Project Management and Team Work}
The Bicycle Management System was developed through effective team collaboration, with clear role distribution across design, frontend, backend and database management. A cost-effective version can be built using open-source technologies and local servers, suitable for educational or small-scale community projects. Revenue generation may come from three streams: subscription models for institutions, advertising or partnerships with local bike vendors. Projected revenue from 1,000+ active users or installations across universities and offices ensures a balanced approach to affordability and scalability, maintaining both social impact and financial sustainability.\cite{4.2}

\section{Complex Engineering Problem}
\textbf{Problem Statement:} Managing bicycle rentals or access in a large-scale system can be complex, especially when tracking multiple users, bicycles and usage history. Traditional manual systems may lead to inefficiencies, confusion and challenges in maintaining real-time data, particularly when managing access, availability and tracking in environments like universities or urban communities.\\
\textbf{The suggested Solution:}
The Bicycle Management System offers a solution by digitally organizing and automating bicycle access through a web-based platform. Users can register, log in and check the availability of bicycles in real time. The system is built using Object-Oriented Programming (OOP) principles, ensuring modularity, scalability and reusability in the codebase.\\
\textbf{Key Features of the Solution:}
\begin{itemize}
\item User Authentication and Role Management using OOP classes to ensure secure access and individualized privileges.
\item Bicycle Tracking with real-time updates, allowing users to view available bicycles based on location and status.
\item Efficient Database Management using OOP methods to interact with MySQL, ensuring smooth data retrieval and updates without redundancies.
\item Admin Control for managing user accounts, bicycle maintenance, and usage history.\cite{4.3}
\end{itemize}

\subsection{Mapping of Program Outcome} 
\textbf{Problem Statement:}
The manual operation of bicycle rental systems, particularly in large institutions or public spaces, can be time-consuming and inefficient. Users may face difficulties in locating available bicycles, while administrators may struggle with managing real-time data, user tracking and system updates. A traditional, non-automated approach can lead to delays and inaccurate information.\\
\textbf{The suggested Solution:}
The Bicycle Management System is a digital solution that automates the bicycle rental process, enabling users to easily locate, reserve, and rent bicycles through a web-based platform. Built with Object-Oriented Programming (OOP) principles, the system offers a modular, scalable design that allows for real-time updates, user authentication, and efficient resource management. By automating the system and using object-oriented principles, the platform simplifies management for both users and administrators.\\
\textbf{Key Features of the Solution:}
\begin{itemize}
\item Real-time data is updated and processed through efficient object interactions, allowing users to view available bicycles.
\item Admin users can monitor and manage bicycles, users, and usage logs, ensuring smooth operations.
\item MySQL is used with OOP methods to store, retrieve and manage data, ensuring smooth and error-free interactions.
\item The design supports both small-scale implementations for local setups and large-scale applications for city-wide bike-sharing systems.\cite{4.3.1}
\end{itemize}

\begin{center}
    \begin{table}[ht]
    
        \begin{tabular}{|p{0.2\textwidth}|p{0.7\textwidth}|}
            \hline
            \textbf{PO's} & \textbf{Justification} \\
            \hline
            PO1 & Demonstrates the application of fundamental software engineering principles, focusing on object-oriented design for creating a scalable, maintainable system. \\
            \hline
            PO2 & Effectively determines the requirements of users (both end-users and administrators) and implements a practical, intuitive solution that meets their needs.\\
            \hline
            PO3 & Designs and develops a functional bicycle management system, using object-oriented programming principles to ensure modularity and long-term maintainability. This system addresses real-world challenges like bicycle availability, tracking and user management.\\
            \hline
        \end{tabular}
        \centering
        \caption{Justification of Program Outcomes.}
        \label{tab:po_justification}
    \end{table}
\end{center}

\begin{center}
    \begin{table}[ht]
    \subsection{Complex Problem Solving} 
        \begin{tabular}{|p{0.12\textwidth}|p{0.12\textwidth}|p{0.12\textwidth}|p{0.12\textwidth}|p{0.12\textwidth}|p{0.12\textwidth}|p{0.12\textwidth}|}
        \hline
        EP1& EP2& EP3& EP4& EP5& EP6& EP7\\
        Dept of Knowledge & Range of Conflicting Requirements & Depth of Analysis & Familiarity of Issues & Extent of Applicable Codes & Extent of Stakeholder Involvement & Inter-dependence\\
        \hline 
         The project requires knowledge of the project involves expertise in OOP, software architecture and database management. Main challenges include ensuring data integrity, real-time updates and a smooth user interface, which are tackled using modular design and OOP techniques.
&The project requires involves balancing data accuracy and system efficiency, minimizing booking errors and availability conflicts. Key skills include programming, database management, algorithm design and object-oriented programming. Effective data processing is critical for optimal real-world performance.
 &The project involves moderate software integration, focusing on reliable system functionality. Key skills include OOP design, database management, API integration and interface development. Balancing cost and performance is essential, addressing challenges like data synchronization and system responsiveness common in software development.
&Testing involves common challenges like debugging software errors and adjusting system logic. The system must balance performance with user-friendliness, being both responsive and reliable in different conditions. Identifying bugs and refining features are crucial for optimal system performance.
&The project involves multiple segments of code that collectively implement the system’s functionality. The extent of applicable codes covers the full structure of the application using Object-Oriented Programming principles, ensuring modularity, reusability and scalability.
&The success of the Bicycle Management System relies heavily on the active involvement of stakeholders throughout the project lifecycle. Stakeholders contribute to defining requirements, guiding development and ensuring the system meets real-world needs.
&In this project involves a network of interrelated modules that must work in harmony. From booking bicycles to tracking returns and calculating costs, each part depends on the smooth functioning of the others, making inter-dependence a core aspect of the system’s design.
\\
        &&&&&&\\
        \hline 
        \end{tabular}
         \centering
        \caption{Mapping with complex problem-solving.}
        \label{tab:p_solve}
    \end{table}
     
\end{center}

\begin{center}
    \begin{table}[ht]
    \subsection{Engineering Activities}
    
        \begin{tabular}{|p{0.18\textwidth}|p{0.18\textwidth}|p{0.18\textwidth}|p{0.18\textwidth}|p{0.18\textwidth}|}
        \hline
        EA1& EA2& EA3& EA4& EA5\\
        Range of resources & Level of Interaction & Innovation & Consequences for society and environment & Familiarity\\
        \hline 
         Simpler resources, such as databases, programming languages (e.g., Java) and development tools, are used. The project requires cooperation between software developers and database administrators. Typical components include user interfaces and backend systems, with creativity focused on designing algorithms for booking management and bicycle availability.
         &Low engagement in component choice as it can be a solo task, though cooperation with database providers or software tools may be needed. Accessibility of development tools (e.g., IDEs, frameworks) is important. While the tools themselves require little creativity, the design of the system architecture and user experience is where the creative effort lies in combining these components into a functional system.

         &Creative algorithms and data handling methods are needed for managing bicycle availability and booking systems. Software tools (e.g., IDEs, database management tools) are used for modeling and implementation, along with physical components like user interfaces. High engagement between frontend design and backend logic is required, as the programming must align with system architecture for efficient functioning.
         &The project emphasizes efficiency and system reliability, which can have positive effects on user experience. Environmental issues are minimal if data privacy practices are followed. Testing tools (e.g., debuggers, unit tests) and a proper testing environment are crucial. Algorithm optimization for bicycle booking and availability management is important for enhancing performance and user satisfaction.
         &This project involves designing a Smart Bicycle Booking System using Object-Oriented Programming, where creative algorithms and effective data handling are essential for managing real-time bicycle availability, bookings and returns.
         \\
        &&&&\\
        \hline 
        \end{tabular}
        \centering
                \caption{Mapping with complex engineering activities.}
        \label{tab:e_act}
    \end{table}
\end{center}
