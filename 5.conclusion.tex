\chapter{Conclusion}

%This is for reference only. Delete before finalization

This chapter presents an overview of the Bicycle Management System project, highlighting the key objectives, the methodology adopted and the results obtained. It addresses the challenges faced during development and proposes future improvements aimed at enhancing the system’s functionality, dependability, and scalability.

%This is for reference only. Delete before finalization

\section{Summary}
\textbf {Summary:} The aim of the Bicycle Management System project is to offer an efficient and user-friendly method for organizing and monitoring bicycle usage through a digital solution. The system incorporates essential elements such as user registration, bicycle availability tracking, booking options, and data management. Tools used include backend services, a database, and a responsive frontend interface. The application was designed to facilitate easy access and control over bicycle-related operations. Although the system successfully performed core tasks like managing bookings and displaying status, certain limitations were faced, such as occasional delays in data updates. Despite these issues, the project demonstrated the practical utility of a technology-driven management system and opens opportunities for further development to enhance performance, stability, and feature expansion.\cite{5.1} 
\section{Limitation}
Even though the Bicycle Management System was implemented successfully, several limitations were observed during development and testing:\\
\textbf{Real-Time Data Sync:} The system occasionally experienced delays in updating availability and booking status, affecting real-time responsiveness.\\
\textbf{Scalability Constraints:} The current implementation is suited for small-scale use and may require architectural adjustments to handle a larger user base or fleet.\\
\textbf{User Authentication:} While basic login functionality exists, enhanced security features such as role-based access and OTP verification were not fully integrated.\\
\textbf{Payment Integration:} The project did not include a fully functional payment method, which limits its practical use for commercial or rental-based systems.\\
\textbf{Database Handling:} Minor issues were observed in managing concurrent access and ensuring data consistency under multiple user interactions.\cite{5.2} \\
\section{Future Work}
Future enhancements aimed at overcoming the current limitations and adding more features to the Bicycle Management System include: \\
\textbf{Payment System Integration:} Implementing a digital payment method will enable secure and seamless transactions, making the system more practical for real-world usage. \\
\textbf{Mobile Accessibility:} Introducing mobile support, either through a responsive app or web version, will offer greater convenience and improve user interaction.\\ 
\textbf{Admin Controls:} Adding more advanced administrative tools and role-specific access will strengthen control over the system, especially for larger operations. \\
\textbf{Notification System:} A built-in alert mechanism, such as email or in-app notifications, can improve communication regarding booking status, return times, or system updates.\cite{5.3.1} \\
\\
In essence, although the Bicycle Management System has achieved its core objectives, addressing the mentioned limitations and incorporating these proposed upgrades will significantly improve its performance, reliability and scalability. The system holds potential for use in university campuses, smart city programs, commercial bike-sharing services and other organized transportation networks.\cite{5.3.2}
